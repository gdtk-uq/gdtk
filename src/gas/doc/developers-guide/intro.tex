\chapter{Introduction}
\label{chap:intro}

The \dlang gas module provides services and data structures that model
the behaviour of a continuum gas.
This module is used primarily as a service module in application codes that require
modelling of complex gases.

This document describes the source code in the gas module.
We will describe what is in the module (the layout of code),
why things are the way they are (the code design), 
and give hints on how to add new models and features.
The module itself was designed in a top-down manner and
it seems logical to organise this descriptive document
in that manner also.
The remainder of this Introduction describes the two
centrepiece structures with which the calling code interacts.
In other words, this first chapter serves as a mini API documentation,
if you will.
The two centrepiece structures are the \type{GasState}  
and the \type{GasModel}.
The remainder of the document is for developers who want
to understand the internals of the module.

\section{Overview of source code layout}

\section{GasState data structure}
\label{sec:GasState}

The \type{GasState}\index{GasState} data structure, as its name implies,
holds data members related to describing the state of the gas.
The data members included \type{GasState} are all set as \type{public}
members so that they may be freely manipulated by the owner of a 
\type{GasState} object. This is a deliberate choice; the \type{GasState}
is meant to function as a simple container for data related to the
gas.
The particular members and their units are shown in the source code
snippet included in Listing~\ref{lst:GasState} which has been extracted
from the source file \type{gasmodel.d}.

\begin{Listing}
\lstinputlisting{listings/gasstate.txt}
\caption{Excerpt from \type{gasmodel.d} showing member data in \type{GasState} struct.}
\label{lst:GasState}
\end{Listing}

If you are wondering why the energy, temperature and thermal conductivity are
vector fields it is because the \type{GasState} is designed to suit a 
general multi-temperature hypersonic flow solver.
Note that we omit certain common thermodynamic values from the \type{GasState} struct such as enthalpy and entropy.
One could also make an argument for including Gibbs free energy, and the list
could go on.
Our choice to limit the number of members in the data struct is motivated
by memory concerns.
The \type{GasState} struct is used many times in out CFD application code:
it is an embedded member of our FlowState object.
Thus, each additional member variable in the \type{GasState} incurs a direct
memory cost proportional to the number of grid points in a CFD mesh.
Our loose criterion for including certain members is based on those values
that are used most frequently by our CFD code.
For all other values, like entropy and enthalpy, we choose to compute them
as needed.
Bear this reasoning in mind if you think the \type{GasState} could benefit
from another data member in the structure.

We provide five constructors for a \type{GasState} object
as shown in Listing~\ref{lst:GasState-cons}.
Ths first requires the caller to explicitly pass
the number of species and number of energy modes so
that the internal arrays can be set.
The next two constructors are both passed a \type{GasModel}
object and get the information about number of species
and modes from that object.
These constructors are the most commonly used from calling
code when a \type{GasModel} has already been initialised.
Additionally these constructors require that an initial
pressure and temperature(s) be given.
The difference between the second and third constructor
is that the \type{T\_init} parameter is an array in the
second constructor, whereas it is a single value in the third.
When using the third constructor for a \type{GasState} with multiple
modes, all modes are initialised to the value \type{T\_init}.
The fourth constructor is a copy constructor.
The final constructor is the postblit constructor.
This is provided so that proper value semantics are behaved 
when one \type{GasState} object is assigned to another.
The postblit constructor correctly provides a deep copy
by making duplicates of the internal arrays.

\begin{Listing}
\lstinputlisting{listings/gasstate-cons.txt}
\caption{Constructors for a \type{GasState} object. Source code from \type{gasmodel.d}.}
\label{lst:GasState-cons}
\end{Listing}

\section{GasModel class}
Objects of type \type{GasModel} are the primary means for interacting
with the gas module.
The \type{GasModel} class is an all-encompassing class that provides
a large number of services to the caller.
Those services include: thermodynamic state and derivative evaluations;
calculation of diffusion coefficients; and providing property information
about the gas model.
The public methods of the \type{GasModel} class are shown in Listing~\ref{lst:GasModel}.
The services provided by the \type{GasModel} are not exhaustive.
However, we have found that they are sufficient for us when building
CFD and chemical kinetic applications.

\begin{Listing}
\lstinputlisting{listings/gasstate-cons.txt}
\caption{Services offered by \type{GasModel} object via its public methods. Source code from \type{gasmodel.d}.}
\label{lst:GasModel}
\end{Listing}

A large number of methods in the \type{GasModel} object are passed a reference
to a \type{GasState} object.
In the \type{void}-returning methods, the caller can expect that some
or all of the values in the \type{GasState} object are modified during
the method call.
For example, the method \type{update\_thermo\_from\_pT} will use the 
values present in the pressure and temperature fields of the
\type{GasState} and from that update the density and energy fields
to a consistent thermodynamic state.
So on return from the call, the density field (\type{rho}) and
the energy vector (\type{e[]}) will have been updated.
The \type{double}-returning methods use the values in
the \type{GasState} object to compute and return the double
associated with the method name.

\subsection{GasModel derived classes}

Mention that a \type{GasModel} is abstract.

Talk about building specific models by composition.






